\documentclass[journal]{IEEEtran}
\usepackage{blindtext}
\usepackage{graphicx}
\usepackage[cmex10]{amsmath}
\usepackage{array}
\usepackage{mdwmath}
\usepackage{mdwtab}
\usepackage{eqparbox}
\usepackage{color} 
\usepackage{tabularx}  
\usepackage{epsfig}
\usepackage{amsmath} 
\usepackage{amssymb} 
\usepackage{bm}
\usepackage{graphicx}
\usepackage{wasysym} 
\usepackage{epsfig}
\usepackage{fancyhdr}
\usepackage{float}

\begin{document}
\title{Pre Lab 4}  
\author{Felix Martinez 502}
\maketitle
\section{\textbf{Introduction}}
This lab will focus on the introduction and experimentation over
basic transistor and op amp circuits. It will be split into two
parts, the first part will compose of comparing the gain of 
a regular High Pass Filter to that of a Transistor Buffer
Circuit design. A High Pass Filter filters out any 
frequencies below a calculated wavelength. However, depending 
on your input voltage, the filtered out frequencies may not 
have enough current to become useful in other electrical 
components. This is why we are adding a transistor, a
transistor takes in a small electrical current and releases a 
larger current output, to hopefully increase the gain of the 
circuit. The second part of the experiment will compose of 
building two types of op amp circuits, the Inverting Op-Amp, 
and the Non Inverting Op-Amp. For this section of the lab we 
will also compare the gain vs. frequency from both of these 
circuits and test if the placement of resistors would affect 
their gain. Gain is the ratio between the input voltage and 
the output voltage, a positive gain would give off a higher 
output voltage than the input, or a gain higher than one, a 
negative gain would give off a lower output voltage than the 
input, giving a value lower than one. This experiment is conducted exclusively on 
Multisim, therefore all of the measurments will be perfect and will amount to no error 
in the final data.

\section{\textbf{Experimental Setup/Diagrams}}
To be clear, for all of the circuit diagrams with the $\Omega$ 
notation, it stands for where I will be taking my measurments on the circuit design 
software, Multisim. This experiment is conducted in AC, therefore the sine notation on the 
circuit diagrams stands for AC Power, and the +/- notation on the circuit diagrams stands for
DC Power, meant to power up the Transistors and Op Amps.

\subsection{Transistor vs. Filter} 
\begin{figure}[H]
    \centering
    \includegraphics[width=\linewidth]{Filter.PNG}
    \caption{High Pass Filter Circuit. $R_0$ = $8.2$ $k\Omega$, $C_0$ = $2.09$ nF.}
\end{figure}
\begin{figure}[H]
    \centering
    \includegraphics[width=\linewidth]{Transistor_Buffer.PNG}
    \caption{Transistor Buffer Circuit. $R_0$ and $C_0$ are the 
    same values as the \textbf{Figure 1} high pass filter circuit.}
\end{figure}
\subsection{Op-Amps} 
\begin{figure}[H]
    \centering
    \includegraphics[width=\linewidth]{Inverting_opamp.PNG}
    \caption{Inverting Op-Amp circuit.}
\end{figure} 
\begin{figure}[H]
    \centering
    \includegraphics[width=\linewidth]{Noninverting_opamp.PNG}
    \caption{Non Inverting Op-Amp circuit.}
\end{figure}  

\section{\textbf{Procedure}}
\subsection{Transistor vs. Filter}
    Arrange a capacitor and a resistor in series like in 
\textbf{Figure 1}. Find the gain across the resistor by cycling through a large range 
of frequencies to find how the gain changes when the frequency is increased. \newline

$\indent$Next, attach the filter to a transistor with a 
resistor placed in sequence after it, like in \textbf{Figure 2}.  Find the gain across the
resistor after the transistor by cycling through the same range of frequencies that you 
have done for the High Pass Filter. Rotate out this resistor and cycle through the same
range of frequencies with at least 2 other resistors 
with higher and lower values, the values we have chosen in this experiment are outline 
in \textbf{Figure 5}. Finally, 
graph the frequency vs. gain plot for the Transistor Buffer Circuit and the High Pass 
Filter and compare the results.

\subsection{Op-Amps}
Begin by choosing an Op-Amp (in this experiment we will
us the LF411 but any Op-Amp would do) and 2 sets of 2 different
resistors, they will be called $R$ and $R'$. First, start by constructing the Inverting Op-Amp 
circuit as designed in \textbf{Figure 3} using the $R$ set of resistors. Find the expected 
gain of the circuit by using this equation:

\begin{equation}
    G = -\frac{R_1}{R_2}
\end{equation}

Compare this to the gain measurmentes of the circuit by cycling through
a large range of frequencies in Multisim. Swap out 
the resistors by changing their places in the circuit and repeat the procedure. 
Next, bring in the second set, these will be denoted by using $R'$, and repeat the previous
instructions. Finally, plot the gain vs. frequency graph and 
compare the expected to the measured gain for both the 
$R$ and $R'$ sets of resistors. \newline

$\indent$ Next, construct the Non-Inverting Op-Amp circuit as
designed in \textbf{Figure 4}. Find the expected gain of this
circuit by using this equation:

\begin{equation}
    G = 1 + \frac{R_1}{R_2}
\end{equation}

Then compare this to the gain measurmentes of the circuit by cycling through a large
range of frequencies in Multisim. Cyle through the 
resistances as described in the Inverting Op-Amp section, and 
plot the gain vs. frequency graphs for both $R$ and $R'$ and 
compare.

\section{\textbf{Data/Analysis}}
\subsection{Transistor vs. Filter}
    Below is data collected from the High Pass Filter Circuit outlined in 
    \textbf{Figure 1} and the Transistor Buffer Circuits outlined in \textbf{Figure 2}.
    
    \begin{figure}[H]
        \centering
        \includegraphics[width=\linewidth]{Transistor_Filter.PNG}
        \caption{Gain with and without a transistor, with different resistances.}
    \end{figure}

The Gain of the filter circuit caps out at 1, and maintains that gain for every frequency after
it reaches its maximum. The Transistors have a very similar relationship, however, they can only
maintain its maximum gain for a prolonged range, after which, it dips down to a gain of 0.2 
where it then stays for every frequency after the dip. All of the transistor circuits start
their decline uniformly at frequencies of around $100,000,000$ Hz, this is important when 
looking at their inclination. One can see a trend in which an increased resistance results 
in a lower peak frequency, which then results in a longer range at a gain of 1. This also has an impact with lower resistances such as the $500 \Omega$ 
resistor, that due to such a higher cuttoff frequency, it's range where the gain is maxed 
may be small enough to where it may not be able to reach a gain of 1, like the other, 
higher resistors. Finally, a result of powering on the transistor with the DC power supply
does not allow the transistor circuits to fall back down to a gain of zero after the 
transistor gets turned on at the higher frequencies. This is a result of the leftover
current in the circuit from the DC power supply.

\subsection{Op Amps}
    Below are the Resistors that were used when conducting part B of this lab. Both 
    $R_1$ and $R_2$ were used in the circuits outlined in \textbf{Figure 3} 
    and \textbf{Figure 4}.

    \begin{figure}[H]
        \centering
        \includegraphics[width=\linewidth]{Resistors.PNG}
        \caption{$R_1$ and $R_2$ values used in part B of the experiment.}
    \end{figure}

    Below Is the data for the Inverting Op-Amp Circuit as designed in \textbf{Figure 3}.

    \begin{figure}[H]
        \centering
        \includegraphics[width=\linewidth]{Inverting_expected_table.PNG}
        \caption{Expected vs. Measured values of Gain for the Inverting Op-Amp Circuit.}
    \end{figure}

    \begin{figure}[H]
        \centering
        \includegraphics[width=\linewidth]{Inverting_OpAmp_LF411.PNG}
        \caption{Gain Vs Frequency of the LF411 Inverting Op-Amp with Resistors outlined in 
        \textbf{Figure 6}.}
    \end{figure}
    
    $\indent$ For the Inverting Op-Amp the measured order of the gain matches that of the expected
    order, however, the measured values of the gain do not equal the expected values 
    when using the Resistors in \textbf{Figure 6} and plugging their values into equation (1).
    The gain decreases down
    to zero at higher frequencies, this is due to the properties of Op-amps and that the circuit 
    is in a close loop. Normally, Op-amps are very stable up to 1 MHz frequency, but after this
    due to the limitations held by the internal components of the Op-Amp, higher frequencies can begin to 
    oscillate the output lead, causing a ringing in the circuit. To avoid this, many Op-Amps,
    like the LF411 we are using, have an internal capacitor to prevent frequencies above 1 MHz, 
    making a low pass filter.
    This relationship can be seen in the data as the gain of each circuit begins to deminish at
    around 1 MHz frequency until it falls to a gain of zero.\newline

    Below is the data for the Non-Inverting Op-Amp Circuit as designed in \textbf{Figure 4}.

    \begin{figure}[H]
        \centering
        \includegraphics[width=\linewidth]{NonInverting_expected_table.PNG}
        \caption{Expected vs. Measured values of Gain Table for the Non-Inverting Op-Amp}
    \end{figure}

    \begin{figure}[H]
        \centering
        \includegraphics[width=\linewidth]{NonInverting_OpAmp_LF411.PNG}
        \caption{Gain Vs Frequency of the LF411 Non-Inverting Op-Amp with Resistors outlined in 
        \textbf{Figure 6}.}
    \end{figure}
    
    When loooking at the data for the Non-Inverting Op-Amp, the first thing to notice is that
    the measured gain does not change when flipping the $R_1$ and $R_2$ resistors, this characteristic 
    is carried across both the $R$ and $R'$ sets of resistors. Much like the Inverting Op-Amp,
    the measured order of the gain also matches that of expected when comparing the non flipped
    $R$ and $R'$ values in this circuit. However, when 
    plugging the values of $R$ and $R'$ into equation (2) the measured gain values are completely
    off that of the expected values. Furthermore, like the Inverting Op-Amp,
    the Non-Inverting Op-Amp follows the same rules of the gain
    drop as outlined in the previous section, which involve a decrease in the gain beggining at 
    around 1 MHz dropping to a gain of zero at extremly high frequencies.

    \section{\textbf{Conclusion}}
    In \textbf{Part A} of this lab, it was shown that when adding on a transistor to a High Pass 
    Filter circuit, it shortens the range at which its gain is maximized. This relationship is 
    also dependant on what size of a resistor is used after the implimentation of the transistor.
    Resulting with the relationship that lower valued resistors, result in a less amount of time where 
    its gain 
    will be maximized, which, with a low enough resistance, could result in a gain less than $1$ at its highest. In 
    \textbf{Part B} of this lab, it was shown that in Inverting Op-Amp Circuits, the order in 
    which the resistors are placed matters. However, strangely this relationship was not observed
    in the Non-Inverting Op-Amp Circuit. % this could be because...
    Another unexpected reult of this lab was that the measured gain was not the same as the
    expected gain, even though the order in which they fell from greatest to least were the 
    same as expected. % this could be a result of...
    However, this section of the lab was useful in understanding why closed loop Op-Amp circuits
    act simliar to low pass filters, due to the limitations of the components inhabiting most
    Op-Amps, the manufacturers must add on a capacitor to prevent the circuit from oscillating, 
    giving off an unreliable output.
\end{document}
